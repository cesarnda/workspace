\documentclass[tikz, 12pt]{scrartcl}
\usepackage{etex}
\usepackage{geometry}                % See geometry.pdf to learn the layout options. There are lots.
\geometry{letterpaper}                   % ... or a4paper or a5paper or ... 
%\geometry{landscape}                % Activate for for rotated page geometry
%\usepackage[parfill]{parskip}    % Activate to begin paragraphs with an empty line rather than an indent
\usepackage{graphicx}
\usepackage{amssymb}
\usepackage{epstopdf}
\usepackage{amsmath}
\usepackage{subfig}
\usepackage[ruled,vlined, linesnumbered]{algorithm2e}
\usepackage{fancyhdr}
\usepackage{anysize}
\usepackage{vaucanson-g}
\usepackage{amsthm}
\usepackage{longtable}
\usepackage{hyperref}
\usepackage{tikz}
\usetikzlibrary{arrows}

\DeclareGraphicsRule{.tif}{png}{.png}{`convert #1 `dirname #1`/`basename #1 .tif`.png}

\title{Data Structures and Algorithms}
\subtitle{Spring 2014}
\author{Cesar Agustin Garcia Vazquez}
\date{\today}                                           % Activate to display a given date or no date
\allowdisplaybreaks
\begin{document}
\maketitle
\section{Programming Assignment 01}

\subsection{Introduction}
For this assigment, I work with a MacBook Aluminum latex 2008, with a 2 GHz Intel Core 2 Duo processor and 8 GB 1067 MHz DDR3 in RAM. \\
All the experiments run for this assignment are with a complete graph, so generating $|V|(|V| - 1)/2$ edges might not feasible for certain values. The number of nodes the program is expected to handle is shown in Table \ref{cardinalities}, which helps to determine what values can be computed straightforward and what values required special treatment. If we consider $n = 2^k$, then the number of edges is $2^k(2^k - 1) / 2 < 2^k \cdot 2^k / 2 = 2^{2k} / 2 = 2^{2k - 1}$, then we have an upper bound for the number of edges as a power of 2.

\begin{table}[ht!]
\caption{\label{cardinalities}Cardinality of nodes and edges}
\centering
\begin{tabular}{|c|c|c|c|c|}
\hline
Nodes		&			&	Edge			& Upper bound	&	\\
\hline
16			&	$2^4$	& 	120	 		&	128		&	$2^{7}$\\
32			&	$2^5$	&	496			&	512		&	$2^{9}$\\
64			&	$2^6$	&	2,016		& 2,048		&	$2^{11}$\\
128			&	$2^7$	&	8,128		& 8,192		&	$2^{13}$\\
256			&	$2^8$	&	32,640		&32,768		&	$2^{15}$\\
512			&	$2^9$	&	130,816		&131,072		&	$2^{17}$\\
1,024		&	$2^{10}$	& 	523,776		&524,288		&	$2^{19}$\\
2,048		&	$2^{11}$ 	& 	2,096,128		&2,097,152	&	$2^{21}$\\
4,096		&	$2^{12}$	&     8,386,560		&8,388,608	&	$2^{23}$\\
8,192		&	$2^{13}$	&   33,550,336		& 33,554,432	&	$2^{25}$\\
16,384		&	$2^{14}$	& 134,209,536		& 134,217,728	&	$2^{27}$\\
32,678		&	$2^{15}$	& 533,909,503		& 536,870,912	&	$2^{29}$\\
\hline
\end{tabular}
\end{table}

Considering how much RAM I have, we have that 8 GB = 8,192 MB = 8(1,024) MB = 8,388,608 KB = $8(1,024)^2$ KB = 8,589,934,592 B = $8(1,024)^3$ B = $8 (2^{10})^3 B = 8(2^{30})$ B, so this gives an idea of how much I can compute.
\\
if I assign 4,096 MB for memory heap and I tried to create a complete graph with $16,384$ nodes in $\mathbb{R}^4$, I get an OutOfMemoryError while trying to create the edge number 38, 293, 002. If I increase the memory heap to 6,144 MB, then I get the error while creating the edge $50,331,651$. Hence it is not going to be possible to create $134,217,728$ edges and keep all of them at the same time in memory. \\
\\
For the cases $2^{13} + 1$ nodes and more, 
$8,193$ nodes, weight computed from points in $\mathbb{R}^3$, my own MergeSort implementation,  and using  

\begin{table}
\centering
\caption{Running time with different values of $n$}
\begin{tabular}{|c|c|}
\hline
$n$	&	Time in milliseconds	\\
\hline
16		&              0 \\
32		&	          3\\
64		&		   5\\
128		&              9 \\
256		&		239\\
512		&	      436\\
1,024	&	    1,800\\
2,048	&	  14,371\\
4,096	&	  39,714\\
8,192	&	177,060	\\
16,384	&	466,158\\
32,678	&  2,190,921\\
\hline
\end{tabular}
\end{table}

\end{document}  