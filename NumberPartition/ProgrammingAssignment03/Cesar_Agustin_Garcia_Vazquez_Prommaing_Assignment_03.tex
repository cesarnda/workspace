\documentclass[tikz, 12pt]{scrartcl}
\usepackage{etex}
\usepackage{geometry}                % See geometry.pdf to learn the layout options. There are lots.
\geometry{letterpaper}                   % ... or a4paper or a5paper or ... 
%\geometry{landscape}                % Activate for for rotated page geometry
%\usepackage[parfill]{parskip}    % Activate to begin paragraphs with an empty line rather than an indent
\usepackage{graphicx}
\usepackage{amssymb}
\usepackage{epstopdf}
\usepackage{amsmath}
\usepackage{subfig}
\usepackage[ruled,vlined, linesnumbered]{algorithm2e}
\usepackage{fancyhdr}
\usepackage{anysize}
\usepackage{vaucanson-g}
\usepackage{amsthm}
\usepackage{longtable}
\usepackage{hyperref}
\usepackage{tikz}
\usepackage{tkz-fct}
\usetikzlibrary{arrows}
\usepackage{pgfplots}
\usepackage{listings}
\usepackage{color}
\usepackage{lscape}

\DeclareGraphicsRule{.tif}{png}{.png}{`convert #1 `dirname #1`/`basename #1 .tif`.png}

\title{Data Structures and Algorithms}
\subtitle{Spring 2014}
\author{Cesar Agustin Garcia Vazquez}
\date{\today}                                           % Activate to display a given date or no date
\allowdisplaybreaks
\begin{document}
\maketitle
\section{Programming Assignment 03}

\subsection{Introduction}

For this programming assignment, you will implement a number of heuristics for solving the NUMBER PARTITION problem, which is (of course) NP-complete. As input, the number partition problem takes a sequence $A = (a_1, a_2, \ldots, a_n)$ of non-negative integers. The output is a sequence $S = (s_1, s_2, \ldots, s_n)$ of signs $s_i \in \{ -1, +1 \}$ such that the \textit{residue}, is as shown in equation (\ref{residue}), is minimized.

\begin{equation}\label{residue}
u = \left| \sum_{i = 1}^n s_i \cdot a_i \right|
\end{equation}

Another way to view the problem is the goal is to split the set (or multi-set) of numbers given by $A$ into two subsets $A_1$ and $A_2$ which roughly equal sums. The absolute value of the difference of the sums is the residue.\\
\\
In this programming assignment, we are going to analyze and implement several heuristic methods to solve this problem.

\subsection{Number Partition in $O(n \cdot b)$}

As a warm-up exercise, you will first prove that even though Number Partition is NP-complete, it can be solved in pseudo-polynomial time. That is, suppose the sequence of terms in $A$ sum up to some number $b$. Then each of the numbers in $A$ has at most $\log b$ bits, so a polynomial time algorithm would take time polynomial in $n \cdot \log b$. Instead you should find a dynamic programming algorithm that takes time polynomial in $n \cdot b$. \\
\\
If we consider sum over the elements in $A$ is $b$, then the program can be reduced as from a set of integers, find the subset whose elements sum $b$. To solve this problem using dynamic program, we are going to consider smaller problems which are given by finding the subset whose elements sum $b - 1$. If we continue this way simplifying this problem, we are going to reach to the simple problem of finding a subset over $n$ elements, whose sum is 1. \\
This problem can be simplified even more if we consider a smaller subset, i.e., finding a subset over $n- 1$ elements, whose sum is 1. If we continue reducing this problem, then we are going to reach the simple problem which is determine if we the first element, is equal to 1. \\
If we can solve this problem, we can solve the problem of the residue by finding the closest element to $b / 2$ that can be yielded by summing a subset of $A$. For this, we are going to create a tableau with $n$ rows, which are going to represent the elements of the set and $b$ columns, that are going to represent the numbers that can be yielded by summing up a subset from $A$. So we have to create a n $n \cdot b$ boolean tableau $D$, in which each entry $(i,j)$ is true if it is possible to sum take the elements in the set $A$ from $a_1$ to $a_j$, i.e., from the first one, up to the $j$-th element.\\ 
The recurrence is going to be given by considering how can $D(i,j)$ be computed if I already know if $D(i - 1, j)$ and $D(i - 1, j- 1)$. If we know that $D(i - 1, j)$ is true, that means that we can yield the number $j$ wit the previous $i - 1$ elements, and we can do it with the first $i$ elements just by not considering the $i$-th element in the subset. \\ 
\\
If $D(i-1, j)$ is false, then it means that so far we have not been able to generate a subset whose elements sum $j$. Hence, we have to consider the information that $D(i - 1, j - 1)$ can give, which only tells us if the number $j- 1$ was able to be computed with $i -1$ elements. If we were to add the element $a_i$ to the set, we would be able to yield the value $j$ as long as the current sum is $j - a_i$. Therefore if $D(i - 1, j - a_i)$ is true, then $D(i, j)$ is true because we just need to add $a_i$ to the subset in $D(i - 1, j - a_1)$.\\
\\
Before applying the recurrence formula in a nested loop, we need to consider a base case or a way to initialize the tableau with a sum that can be yielded. This initialization is going to be given by the minimum element in the array. This algorithm is shown in Algorithm \ref{isSumPossible}.

\begin{algorithm}[th!]
	{\bf funcion} populateTableau(Array $A$)\\
	Long $b$ = $\sum_{a \in A}a$\\
	Long minimum = A.minimum\\
	boolean[] tableau = new boolean[$A.size$][$b$ + 1]\\
	tableau[0][minimum] = \textbf{true} \tcp{initialize tableau} 
	\For{(int $i$ = 1; $i$ $< A.size$; $i$++)}{
		\For{(long $j$ = 1; $j$ $\leq b$; $j$++)}{
			\If{tableau[$i$ - 1][$j$] \textbf{or} tableau[$i$ - 1][$j$ - A[$i$]]}{
				tableau[$i$][$j$] = \textbf{true};
			}
		}
	}
\caption{Algorithm to populate the tableau to check if a sum is possible}
\label{isSumPossible}
\end{algorithm}

So far we can only decide if based on the set $A$, we are able to compute a sum $k$, where $k$ can be $b$ or $b/2$. We want to partition $A$ into $A_1$ and $A_2$, such that if $S_1 = \sum_{a \in A_1} a$ and $S_2 = \sum_{a^{\prime} \in A_2} a^{\prime}$, then $|S_1 - S_2|$ is minimized. Assuming the best case is when $S_1 = m$, then $S_2 = b - m$, then the residue is going to be given by equation (\ref{numberPartitionResidueFormula}).

\begin{eqnarray}
|S_1 - S_2|	&	=	&	|\sum_{a \in A_1} - \sum_{a^{\prime} \in A_2} a^{\prime}| \nonumber \\
				&	=	&	|m - (b - m)| \nonumber \\
				&	=	&	|m - b + m| \nonumber \\
				&	=	&	|2m - b| \nonumber \\
				&	=	&	|2m - 2 \frac{b}{2}| \nonumber \\
				&	=	&	\left|2\left(m - \frac{b}{2} \right)\right| \nonumber \\
				&	=	&	2\left|m - \frac{b}{2}\right| \label{numberPartitionResidueFormula}
\end{eqnarray}

The number partition problem now can be reduced to finding the element $m$ that is possible to yield and is the closest to $b/2$, since the perfect scenario is when $S_1$ and $S_2$ are both equal to $b/2$. So we only have to iterate over the columns from $b/2$ down to 1, until we find that $D(i,j)$ is true. This search might be in $O(n \cdot b)$ but can be reduced if maintain an boolean array $C$ of size equals to $b$, in which an element $c$ is going to be true as long as $D(x, c)$ is true for any $x \in [1, \ldots n]$. The algorithm \ref{isSumPossibleUpdate} shows the updated version to add the new array to keep track of the values that be generated. Notice that so far the complexity of the algorithm is $O(n \cdot b)$, which might be considered as the initialization and population of the tableau.

\begin{algorithm}[th!]
	{\bf funcion} populateTableau(Array $A$)\\
	Long $b$ = $\sum_{a \in A}a$\\
	Long minimum = A.minimum\\
	boolean[][] tableau = new boolean[$A.size$][$b$ + 1]\\
	boolean[] C = new boolean[$b$ + 1]\\
	tableau[0][minimum] = \textbf{true} \tcp{initialize tableau} 
	\For{(int $i$ = 1; $i$ $< A.size$; $i$++)}{
		\For{(long $j$ = 1; $j$ $\leq b$; $j$++)}{
			\If{tableau[$i$ - 1][$j$] \textbf{or} tableau[$i$ - 1][$j$ - A[$i$]]}{
				tableau[$i$][$j$] = \textbf{true};\\
				C[$j$] = \textbf{true}
			}
		}
	}
\caption{Algorithm to populate the tableau to check if a sum is possible in $O(b)$}
\label{isSumPossibleUpdate}
\end{algorithm}

Finally, we just need to find $m$ such that $|m - b / 2$ is minimized. This can be achieved in $O(b)$ by considering just the first value that can be computed starting from $b /2$ down to 1. The algorithm \ref{checkIfSumIsPossible} shows how this search can be done and returns the absolute difference $|m - b/2|$.

\begin{algorithm}[th!]
	{\bf funcion} findSumThatMinimizes(Long b)\\
	Long $b$ = $\sum_{a \in A}a$\\
	\For{(long $m = b /2$; $m > 0$; $m = m - 1$)}{
		\If{C[m] == \textbf{true}}{
			\Return $|m - b / 2|$
		}
	}
	\Return \textbf{null}
\caption{Algorithm check if a sum is possible in $O(b)$}
\label{checkIfSumIsPossible}
\end{algorithm}

Based on equation (\ref{numberPartitionResidueFormula}), the returned value  just needs to be multiplied by 2 and we would have gotten the minimum residue possible. The only thing that we have to consider is when $b$ is odd, which makes $b/2$ be not an integer. For this scenario we just need to return $2 |m - b / 2| + b \mbox{ mod } 2$, to add the missing 1, due to the division.\\
\\
The algorithm works in theory and in practice is going to work fine with small integers, but when we consider integers in the range of $10^{12}$, then creating the tableau and populating it is going to take too long. Also languages like Java do not allow the creating of arrays that big. To solve this problem, we can create a class called \textit{SparseMatrix} that is going to manage big matrices.\\
For this approach we considered an array of maps to store the values that are true for each column, and also an map \textit{MC} to represent the array $C$. The map \textit{MC} is going store only true for the columns $j$ such that $D(x,j)$ is true.\\
We can tune the algorithm a little more by dropping the elements that are larger than $b/2$, which would help reduce the size of the tableau. Also when moving through the columns, we are not considering the case where $S_1 = \{ \}$, this scenario should be validated before, so we can start populating the columns always from the minimum element in the set $A$.\\
Another tunning would be to reduce the size of the tableau from $n \cdot b$ to $n \cdot \lceil \frac{b}{2} \rceil$, since we do not require the values from $\lceil \frac{b}{2} \rceil$.\\
\\
Populating the tableau requires a time in $O(n \cdot b)$ and getting the minimum element is in $O(b)$, so the total running time is $O(n  \cdot b)$.

\subsection{Karmarkar-Karp Algorithm}

One deterministic heuristic for the Number Partition problem is the Karmarkar-Karp algorithm, or the KK algorithm. This approach uses \textit{differencing}. The differencing idea is to take two elements from $A$, call them $a_i$ and $a_j$, and replace the larger by $|a_i - a_j|$ while replacing the smaller by 0. The intuition is that if we decided to put $a_i$ and $a_j$ in different sets, then it is as though we have one element of size $|a_i - a_j|$ around. An algorithm based on differencing repeatedly takes two elements from $A$ and performs a differencing until there is only one element left; this element equals an attainable residue. (A sequence of signs $s_i$ that yields this residue can be determined from the differencing operations performed in linear time by two-coloring the graph $(A, E)$ that arises, where $E$ is the set of pairs $(a_i, a_j)$ that are used in the differencing steps. You will not need to construct the $s_i$ for this assignment.)\\
\\
For the Karmarkar-Karp algorithm suggests repeatedly taking the largest two elements remaining in $A$ at each step and differencing them. For example, if $A$ is initially $(10, 8, 7, 6, 5)$, then the KK algorithm proceeds as in equation (\ref{exampleKK}).

\begin{equation}\label{exampleKK}
\centering
\begin{array}{ccc}
(10, 8, 7, 6, 5)	&	\rightarrow &	(2, 0, 7, 6, 5) \\
				&	\rightarrow	&	(2, 0, 1, 0, 5) \\
				&	\rightarrow &	(0, 0, 1, 0, 3) \\
				&	\rightarrow &	(0, 0, 0, 0, 2)
\end{array}
\end{equation}

Hence the KK algorithm returns a residue of 2. The best possible residue for the example is 0.\\
\\
The fact that we are required to take the largest two elements in each step gives a hint of a sorting involved. If we have $n$ elements, then sorting is going to take $O(n \log n)$. After each sorting, we remove the two largest element and insert the new nonzero value. If we have a sorted linked list, the insertion is going to be in $O(n)$, and for each step, we remove 2 elements and we add 1, so we are reducing the set by 1. Hence, we have the recurrence $T(n) = T(n - 1) + n + 1$ , which considers at most $n - 2$ comparisons to find the sorted place for the new element, as well as retrieving the two largest elements from the sorted linked list, which is done in 2 operations and another operation for the difference. The recurrence 
can be solved by the characteristic equation method, considering the polynomial in equation (\ref{naiveKK}).

\begin{equation}\label{naiveKK}
(x - 1)(x - 1)^2 = (x - 1)^3
\end{equation}

The polynomial in (\ref{naiveKK}) has only one root of multiplicity 3, so the proposed solution for the recurrence is shown in equation (\ref{proposedSolutionForNaiveKK}).

\begin{equation}\label{proposedSolutionForNaiveKK}
T(n)	=	c_1 + c_2 n + c_3 n^2
\end{equation}

Considering the base case for $T(1) = 1$, which just returns the only element in the set, then $T(2) = 4$ and $T(3) = 8$ and the general solution (\ref{proposedSolutionForNaiveKK}), we get the linear system of equations in equation (\ref{linearSystemForNaiveKK}).

\begin{equation}\label{linearSystemForNaiveKK}
\begin{array}{ccccccc}
c_1		&	+	&	c_2		&	+	&	c_3		&	=	&	1\\
c_1		&	+	&	2c_2	&	+	&	4c_3	&	=	&	4\\
c_1		&	+	&	3c_2	&	+	&	9c_3	&	=	&	8
\end{array}
\end{equation}

The linear system in equation (\ref{linearSystemForNaiveKK}) can be solved using Gauss as shown in (\ref{stepsToSolveSystemForNaiveKK}).

\begin{equation}\label{stepsToSolveSystemForNaiveKK}
			\begin{split}
				\begin{array}{cccccc}
					\left(
					\begin{array}{ccc|c}
						1	&	1	&	1	&	1 \\
						1	&	2	&	4	&	4 \\
						1	&	3	&	9	&	8 \\
					\end{array}
					\right)
					&
					\begin{array}{c}
									\\
					R_2 - R_1		 \\
					R_3 - R_1		\\
					\end{array}
					&
					\begin{array}{c}
					 \\
					 \sim\\
					 \\
					\end{array}
					&
					\left(
					\begin{array}{ccc|c}
						1	&	1	&	1	&	1 \\
						0	&	1	&	3	&	3 \\
						0	&	2	&	8	&	7 \\
					\end{array}
					\right)
					&
					\begin{array}{c}
								\\
								 \\
					R_3 - 2R_2	\\
					\end{array}
					&
					\begin{array}{c}
					 \\
					 \sim\\
					 \\
					\end{array}
					\\
					\\
					% Second row
					\left(
					\begin{array}{ccc|c}
						1	&	1	&	1	&	1 \\
						0	&	1	&	3	&	3 \\
						0	&	0	&	2	&	1 \\
					\end{array}
					\right)
					&
					\begin{array}{c}
									\\
								 \\
					\frac{1}{2}R_3 		\\
					\end{array}
					&
					\begin{array}{c}
					 \\
					 \sim\\
					 \\
					\end{array}
					&
					\left(
					\begin{array}{ccc|c}
						1	&	1	&	1	&	1 \\
						0	&	1	&	3	&	3 \\
						0	&	0	&	1	&	\frac{1}{2}\\
					\end{array}
					\right)
					&
					\begin{array}{c}
								\\
					R_2 - 3R_3			 \\
						\\
					\end{array}
					&
					\begin{array}{c}
					 \\
					 \sim\\
					 \\
					\end{array}
					\\
					\\
					% third row
					\left(
					\begin{array}{ccc|c}
						1	&	1	&	1	&	1 \\
						0	&	1	&	0	&	\frac{3}{2} \\
						0	&	0	&	1	&	\frac{1}{2} \\
					\end{array}
					\right)
					&
					\begin{array}{c}
					R_1 - R_3				\\
								 \\
							\\
					\end{array}
					&
					\begin{array}{c}
					 \\
					 \sim\\
					 \\
					\end{array}
					&
					\left(
					\begin{array}{ccc|c}
						1	&	1	&	0	&	\frac{1}{2} \\
						0	&	1	&	0	&	\frac{3}{2} \\
						0	&	0	&	1	&	\frac{1}{2}\\
					\end{array}
					\right)
					&\begin{array}{c}
					R_1 - R_2			\\
								 \\
							\\
					\end{array}
										&
					\begin{array}{c}
					 \\
					 \sim\\
					 \\
					\end{array}
					\\
					\\
					\left(
					\begin{array}{ccc|c}
						1	&	0	&	0	&	-1 \\
						0	&	1	&	0	&	\frac{3}{2} \\
						0	&	0	&	1	&	\frac{1}{2}\\
					\end{array}
					\right)
					\\
				\end{array}
				\end{split}
			\end{equation}

Based on the reduced form yielded in (\ref{stepsToSolveSystemForNaiveKK}), we get that $c_1 = -1$, $c_2 = 3/2$ and $c_3 = 1/2$, which means that the recurrence is given exactly by equation (\ref{exactlyNaiveKK}).

\begin{equation}\label{exactlyNaiveKK}
T(n) = \frac{1}{2} n^2 + \frac{3}{2} n - 1
\end{equation}

We proceed to compute the first 3 values of (\ref{exactlyNaiveKK}) to verify that they match $T(1)$, $T(2)$ and $T(3)$, which is shown in equations (\ref{verifyNaiveKKT1}), (\ref{verifyNaiveKKT2}) and (\ref{verifyNaiveKKT3})

\begin{equation}\label{verifyNaiveKKT1}
\begin{array}{ccc}
T(1)	&	=	&	\frac{1}{2}(1)^2 + \frac{3}{2}(1) - 1 \\	
		&	=	& 	\frac{1}{2} + \frac{3}{2} - \frac{2}{2} \\
		&	=	&	\frac{2}{2} \\
		&	=	&	1
\end{array}
\end{equation}

\begin{equation}\label{verifyNaiveKKT2}
\begin{array}{ccc}
T(2)	&	=	&	\frac{1}{2}(2)^2 + \frac{3}{2}(2) - 1 \\	
		&	=	& 	\frac{1}{2}(4) + \frac{6}{2} - \frac{2}{2} \\
		&	=	& 	\frac{4}{2} + \frac{6}{2} - \frac{2}{2} \\
		&	=	&	\frac{8}{2} \\
		&	=	&	4
\end{array}
\end{equation}

\begin{equation}\label{verifyNaiveKKT3}
\begin{array}{ccc}
T(3)	&	=	&	\frac{1}{2}(3)^2 + \frac{3}{2}(3) - 1 \\	
		&	=	& 	\frac{1}{2}(9) + \frac{9}{2} - \frac{2}{2} \\
		&	=	& 	\frac{9}{2} + \frac{9}{2} - \frac{2}{2} \\
		&	=	&	\frac{16}{2} \\
		&	=	&	8
\end{array}
\end{equation}

As induction hypothesis, we assume that the solution is true, so we have to prove that the recurrence equation is also valid for $n + 1$, which is done in equation (\ref{proofNaiveKK}).

\begin{equation}\label{proofNaiveKK}
\begin{array}{ccl}
T(n + 1) 	&	= 	&	T(n) + n + 2 \\
			&	=	&	\frac{1}{2}n^2 + \frac{3}{2}n - 1 + n + 2\\
			&	=	&	\frac{1}{2}n^2 + n + \frac{1}{2} + \frac{3}{2}n + \frac{3}{2} - 1 \\
			&	=	&	\frac{1}{2}(n^2 + 2n + 1) + \frac{3}{2}(n + 1) - 1 \\
			&	=	&	\frac{1}{2}(n + 1)^2 + 	\frac{3}{2}(n + 1) - 1
\end{array}
\end{equation}

Therefore, the suggested approach has a running time of $O(n^2)$.\\
\\
The algorithm in each step takes two elements and yields another 2 but we are going to consider only the nonzero, so we can place back into the set $A$ the nonzero element, reducing the size of the set by 1. If we want to reduce the running time of the algorithm, we can consider the general recurrence $T(n) = T(n - 1) + O(f(n))$, where $O(f(n))$ is the running time for deleting the two largest elements and inserting the new one. \\
Using a linked list, $f(n) = n$, which means that we must use a data structure which allows to delete an element and insert a new one in running time better than $O(n)$.\\
\\
The best that we can do is to have $f(n) = c$, so we could have that the recurrence will be $T(n) = T(n - 1) + O(1)$, which can be solved considering the polynomial $(x-1)^2$, and hence we have that $T(n) = c_1 + c_2 n$. The base case is still $T(1) = 1$, when we have only one element, and when we have $T(2) = c + 1$, then we have the linear system shown in equation (\ref{bestKKLinearSystem}).

\begin{equation}\label{bestKKLinearSystem}
\begin{array}{ccccccl}
T(1)	&	=	&	c_1 &	+	& 	c_2 	&	=	&	1\\
T(2)	&	=	&	c_1	&	+	&	2c_2	&	=	&	c +1
\end{array}
\end{equation}

If we consider $c_1 = 1 - c_2$, then we have that $1 - c_2 + 2 c_2= c + 1$, from which we get that $c_2 = c$ and $c_1 = 1 - c$. Therefore the recurrence has the solution $T(n) = c \cdot n + 1 - c$, and we can prove it by considering the base case $T(1) = c(1) + 1 - c = c + 1 - c = 1$ and then assuming that it is true for $n$. For inductive step, we have to prove that $T(n + 1) = c \cdot n + 1$, which is done in equation (\ref{inductiveStepBestKK}).

\begin{equation}\label{inductiveStepBestKK}
\begin{array}{ccc}
T(n + 1)	&	=	&	T(n) + c \\
			&	=	&	c \cdot n + 1 - c + c \\
			&	=	&	c \cdot n + 1
\end{array}
\end{equation}

Therefore, the best theoretical running time for the Karmarkar-Karp algorithm is $O(n)$. This theoretical running time can be achieve if we are able to design a data structure that can be built in $O(n)$ and the operations findMax, insert and delete are all of them in $O(1)$. This bound can be achieved with count sort (pigeonhole sort) as long as the values of the elements are in the size of $O(n)$. \\
We first iterate over the elements to find the maximum value, which is done in $O(n)$, then we create an array $A$ of elements of dimension $n + 1$ to include the number $0$. Then we iterate over each element $i$ and then increase its corresponding  bucket in array $A$, as $A[i] = A[i] + 1$; this also takes $O(n)$. The way usually this algorithm is used, the elements in $A$ are placed again in the original array so we waste unnecessary space. We are going to continue using the array $A$, since after each step of the Karmarkar-Karp algorithm, a new element is inserted into the array. The insertion is going to be in $O(1)$ as well as deletion with the operation $A[i] = A[i] - 1$.\\
The only thing that remains is the findMax operation in $O(1)$. In the worst case, this is going to take $O(n)$, which takes us to our original with case of $f(n)$ in $O(n)$. We can amortized the cost of findMax by considering that the size of the array is in $O(n)$, i.e., at the end of the Karmarkar-Karp algorithm, we would have traverse the whole array by assigning a constant cost for each findMax operation equal to the constant $k$ which bounds the size of the array to less than or equal to $k \cdot n$. Therefore, we have the Karmarkar-Karp algorithm running in $O(n)$, which is also in $O(n \log n)$.\\
\\
In case that the size of the integers is not in $(n)$, then we would need to use a different approach. Since we require the largest two elements, this might involve sorting in $\Omega(n \log n)$ or building a heap in $O(n \log n)$. With either option, the operation findMax can be achieve in $O(1)$ but insertion takes $O(\log n)$ with a binary heap and $O(n)$ if the sorted elements are stored a linked list. If the sorted elements are stored in an AVL three the insertion and deletion run in $O(\log n)$ but findMax will also be in $O(\log n)$. If we choose to use a heap to store the elements, then we will have that $f(n) = 3 \log n + 3$, because 2 deletions, 2 findMax and 1 insertion. This yields the recurrence $T(n) = T(n - 1) + 3 \log n + 3$ that can be solved by expanding the recurrence until we reach a general case as in equation (\ref{generalRecurrenceKK}).

\begin{eqnarray}
T(n)	&	=		&	T(n - 1) + 3 \log n + 3 \nonumber \\
		&	=		&	T(n - 2) + 3 \log(n - 1) + 3 + 3\log n + 3 \nonumber \\
		&	=		&	T(n - 2) + 3 \log(n - 1) + 3\log n + 2(3) \nonumber \\
		&	=		&	T(n - 3) + 3 \log(n - 2) + 3 + 3 \log(n - 1) + 3\log n + 2(3) \nonumber \\
		&	=		&	T(n - 3) + 3 \log(n - 2) + 3 \log(n - 1) + 3\log n + 3 (3) \nonumber \\
		&	=		&	T(n - 4) + 3 \log(n - 3) + 3  + 3 \log(n - 2) + 3 \log(n - 1) + 3\log n + 3 (3) \nonumber \\
		&	=		&	T(n - 4)	 + 3\log(n - 3)  + 3 \log(n - 2) + 3 \log(n - 1) + 3\log n + 4 (3) \nonumber \\
		&	\vdots	& \nonumber \\
		&	=		&	T(n - i) + 3 \sum_{k = 0}^{i-1} \log(n - k)  + 3i \label{generalRecurrenceKK}
\end{eqnarray}

If we consider $i = n - 1$, we would have that $T(n) = T(1) +  3 \sum_{k = 0}^{n-2} \log(n - k)  + 3(n - 1)$, which can be expressed as $T(n) = T(1) +  3 \sum_{k = 2}^{n} \log(k)  + 3(n - 1)$. By previous analysis we know that $\sum_{k = 2}^{n} \log(k) $ is in $\Theta(n \log n)$, which implies that using a heap structure, the Karmarkar-Karp algorithm runs in $O(n \log n)$.

\subsection{Karmarkar-Karp algorithm and heuristics}

You will compare the Karmarkar-Karp algorithm and a variety of randomized heuristic algorithms on random input sets. Let us first discuss two ways to represent to the problem and the state space based on these representations. Then we discuss heuristics search algorithms you will use.\\
\\
The standard representation of a solution is simply as a sequence $S$ of +1 and -1 values. A random solution can be obtained by generating a random sequence  of $n$ such values. Thinking of all possible solutions as a state space, a natural way to define neighbors of a solution $S$ is as the set of all solutions that differ from $S$ in either one or two places. This has a natural interpretation if we think of the +1 and -1 values as determining two subsets $A_1$ and $A_2$ of $A$. Moving from $S$ to a neighbor is accomplished either by moving one or two elements from $A_1$ to $A_2$, or moving one or two elements from $A_2$ to $A_1$, or swapping a pair of elements where one is in $A_1$ and one is in $A_2$.\\
\\
A \textit{random move} on this state space can be defined as follows. Choose two random indices $i$ and $j$ from $[1, \ldots, n]$ with $i \neq j$. Set $s_i$ to $-s_i$ and with probability $1/2$, set $s_j$ to $-s_j$.\\
\\
An alternative way to represent a solution called \textit{prepartitioning} is as follows. We represent a solution by a sequence $P = \{p_1, p_2, \ldots, p_n \}$ where $p_i \in \{1, \ldots, n \}$. The sequence $P$ represents a repartitioning of the elements of $A$, in the following way: if $p_i = p_j$, then we enforce the restriction that $a_i$ and $a_j$ have the same sign. Equivalently, if $p_i = p_j$, then $a_i$ and $a_j$ both lie in the same subset, either $A_1$ or $A_2$. \\
\\
We turn a solution of this form into a solution in the standard form using two steps:
\begin{enumerate}
	\item We derive a new sequence $A^{\prime}$ from $A$ which enforces the prepartitioning from $P$. Essentially $A^{\prime}$ is derived by resetting $a_i$ to be the sum of all values $j$ with $p_j = i$, using for example the  pseudocode in algorithm \ref{deriveA}.

\begin{algorithm}[th!]
	$A^{\prime} = (0, 0, \ldots, 0)$\\
	\For{$j = 1$ \textbf{to} $n$}{
		$a_{p_j}^{\prime} = a_{p_j}^{\prime} + a_j$
	}
\caption{Pseudocode to derive $A^{\prime}$}
\label{deriveA}
\end{algorithm}	
	\item We run the KK heuristic algorithm on the result $A^{\prime}$.
\end{enumerate}

For example, if $A$ is initially (10, 8, 7, 6, 4), the solution $P = (1, 2, 2, 4, 5)$ corresponds to the following run of the KK algorithm:

\begin{eqnarray}
A = (10, 8, 7, 6, 5)	&	\rightarrow	&	A^{\prime} = (10, 15, 0, 6, 5) \nonumber \\
(10, 15, 0, 6, 5)	&	\rightarrow	&	(0, 5, 0, 6, 5) \nonumber \\
					&	\rightarrow	&	(0, 0, 0, 1, 5) \nonumber \\
					&	\rightarrow	&	(0, 0, 0, 0, 4)
\end{eqnarray}

Hence in this case the solution $P$ has a residue of 4.\\
\\
Notice that all possible solution sequences $S$ can be regenerated using this prepartition representation, as any split of $A$ into sets $A_1$ and $A_2$ can be obtained by initially assigning $p_i$ to 1 for all $a_i \in A_1$ and similarly assigning $p_i$ to 2 for all $a_i \in A_2$.\\
\\
A random solution can be obtained by generating a sequence of $n$ values in the range $[1, \ldots, n]$ and using this for $P$. Thinking of all possible solutions as a state space, a natural way to define neighbors for a solution $P$ is as the set of all solutions that differ from $P$ in just one place. The interpretation is that we change the repartitioning by changing the partition of one element. A \textit{random move} on this state space can be defined as follows. Choose two random indices $i$ and $j$ from $[1, \ldots, n]$ with $p_i \neq j$ and set $p_i$ to $j$.\\
\\
You will try each of the following three algorithms for both representations.

\begin{itemize}
	\item \textit{Repeated random:} Repeatedly generate random solutions to the problem, as determined by the representations, as seen in the pseudocode in algorithm \ref{repeatedRandom}
	
	\begin{algorithm}[th!]
	Start with a random solution $S$\\
	\For{iter = 1 \textbf{to} max\_iter}{
		$S^{\prime} = $ a random solution\\
		\If{ residue($S^{\prime}$) $<$ residue($S$)) }{
			$S = S^{\prime}$
		}
	}
	\Return $S$
\caption{Repeated random}
\label{repeatedRandom}
\end{algorithm}

\item \textit{Hill climbing:} Generate a random solution to the problem, and then attempt to improve it through moves to better neighbors. The pseudocode can be seen in algorithm \ref{hillClimbing}.

	\begin{algorithm}[th!]
	Start with a random solution $S$\\
	\For{iter = 1 \textbf{to} max\_iter}{
		$S^{\prime} = $ a random neighbor of $S$\\
		\If{residue($S^{\prime}$ $<$ residue($S$))}{
			$S = S^{\prime}$
		}
	}
	\Return $S$
\caption{Hill climbing}
\label{hillClimbing}
\end{algorithm}

\item \textit{Simulated annealing:} Generate a random solution to the problem, and then attempt to improve it through moves to neighbors, that are not always better. The pseudocode can be seen in algorithm \ref{simulatedAnnealing}.

	\begin{algorithm}[th!]
	Start with a random solution $S$\\
	$S^{\prime \prime} = S$\\
	\For{iter = 1 \textbf{to} max\_iter}{
		$S^{\prime} = $ a random neighbor of $S$\\
		\If{residue($S^{\prime}$) $<$ residue($S$)}{
			$S = S^{\prime}$
		}
		\Else{
			$S = S^{\prime}$ with probability $e^{-\frac{\mbox{residue}(S^{\prime}) - \mbox{residue}(S)}{\mbox{T(iter)}}}$
		}
	}
	\Return $S$
\caption{Simulated annealing}
\label{simulatedAnnealing}
\end{algorithm}

Note that for simulated annealing, we have the code return the best solution seen thus far.
\end{itemize}

You will run experiments on sets of 100 integers, with each integer being a random number chosen uniformly from the range $[1, \ldots, 10^{12}]$. Note that these are big numbers. You should use 64 bit integers. Pay attention to things like whether your random number generator works on ranges this large!\\
\\
Now we proceed to generate 50 random instances of the problem as described below. For each instance, find the result from using the Karmarkar-Karp algorithm. Also, for each instance, run a repeated random, a hill climbing, and a simulating annealing algorithm, using both representations, each for at least 25,000 iterations. We are going to give tables and/or graphs clearly demonstrating the results, giving both the numerical results, and the time taken by the algorithms. We are going to compare the results and discuss.\\
\\
For the simulated annealing algorithm, you must choose a \textit{cooling schedule}. That is, you must choose a function T(tier). We suggest T(iter) = $10^{10}(0.8)^{\lfloor \mbox{tier} / 300\rfloor}$ for numbers in the range $[1, \ldots 10^{12}]$, but you can experiment with this as you please.\\
\\
Note that, in our random experiments, we began with a random initial starting point.\\
\\
The approach to develop this section is first to generate the random numbers and store them in a file. We are going to generate 50 plain text files and for each file, 100 integers in the range $[1, \ldots, 10^{12}]$. The integers are going to be distinct since having a repeated element would reduce the complexity of the problem from 100 to 98 since we can place one of the repeated elements in $A_1$ and the other in $A_2$.\\ 
\\
The program is developed in Java SE 1.6 on a Macbook late 2008, which presents some constraints due to the size of the integers. The range of the integers are not considered integers in Java, but long integers. This puts a constraint if we would want to also run the experiments using the dynamic programming approach, to validate the results from the heuristics,  since Java cannot create an array whose size is in the range of long integers. \\
The sparse matrix approach might reduce the complexity but it still takes too long since it has to populate a big tableau.\\
\\
Java provides a way to generate a big random integer using the BigInteger class along with the Random class. Also, if we consider all the numbers in $[1, \ldots, 10^{12}]$ as 13 digits number with padding 0s in the left, then a random long integer can be seen as an array of 13 digits in which the first digit can be 0 or 1 and the other 12 digits can take the random values in $[0, 2, \ldots, 9]$.\\
\\
The heap to implement the Karmarkar-Karp heuristic uses the priority queue that is already built in Java. The algorithm was implemented in a recursive way and in each step, the largest two elements are removed from the heap. Only the difference is added to the heap since the 0 value does not add any value to the algorithm. Also if the difference is 0, it will also not be added to the heap.\\
The stop condition is if the heap is empty, which means that the residue found by the algorithm is the minimum possible and it returns 0. If the heap only has one element, then that is the residue and the algorithm finishes.\\
\\
For Random Repeated, we start with two lists  representing $A_1$ and $A_2$ and we iterate over all the elements in $A$ and randomly we select if we put the element $a_i$ in $A_1$ or in $A_2$. For all the algorithms, the list $A$ is sorted first, so we start with possible residue equals to the greatest element in $A$. We get the sum $S_1$ for $A_1$ and the sum $S_2$ for $A_2$ and we the residue, if the residue is less than our possible residue, then we replace it. If the residue is 0, then we stop, otherwise, we continue until we finish iterating and then we return the possible residue.\\
\\
For Hill Climbing we start with a random solution similar to how we randomly generate a solution in Random Repeated. The neighbor is found by randomly generate a number $i$ between 0 and $|A_1|$ and another number $j$ between 0 and $|A_2|$. In our case we iterate until $j$ is not the same as $i$ but it does not matter if $i = j$ since the indices are for two different sets. \\
After the indices have been generating, the $i$-th element in $A_1$ is swap with the $j$-th element in $A_2$ and then the residue is generating by getting the total sums $S_1$ and $S_2$ and then we get the absolute difference. If the neighbor solution is better than the proposed solution, then the proposed solution is equal to the neighbor solution and the minimum variable is updated.\\
\\
For Simulated Annealing, we start with a generated random solution as in Hill Climbing, but besides keeping tracking of the proposed solution and the neighbor solution, we also keep track of another solution $S^{\prime \prime}$.The neighbor is generated as in Hill Climbing but if the neighbor solution is not better than the proposed solution, we move to the neighbor solution as long as the probability based on the \textit{cooling schedule} T(tier) $= 10^{10}(0.8)^{\lfloor \mbox{tier} / 300\rfloor}$.\\
Finally we keep track of the best solution seen so far with $S^{\prime \prime}$. If at some point by moving to the neighbor solution, the residue is better than the residue in $S^{\prime \prime}$, then we replace $S^{\prime \prime}$ with this solution.\\
\\
For the methods that use the Karmarkar-Karp algorithm, we prefix them with \textit{Mixed}. Instead of randomly swapping elements between $A_1$ and $A_2$, we randomly generate a array of size $|A|$, with elements in $[1, \ldots, |A|]$. After the partition is generated, we get an array $A^{\prime}$ and then from the array we create list of non zero elements and we apply the Karmarkar-Karp program that we already have. If we residue that we have is better than the current one (which initially is the largest element in the array), then we replace it, otherwise we continue until the iterations are completed or we reach a residue of 0.\\
\\
For Mixed Hill Climbing, we start with a random partition generated just as in Mixed Random Repeated and then we get a random neighbor by generating a random number $i$ which will be the value to be replaced and then we generate another random integer $j$ such that it is not the same as $S[i]$ and then we assign $j$ to $S[i]$. Notice that $i$ and $j$ are in the range $[1, \ldots, |A|]$. \\
After each random partition is generated, then we apply KK and compare the residue with the current minimum, if the residue is less than the current minimum, then it is replaced, as well as the current partition by the neighbor partition.\\
\\
For Mixed Simulated Annealing, it is like Simulated Annealing but for the partition as in Hill Climbing. \\
\\
To run the experiments, several test cases were written. The methods that generated the same value for the same input, have test cases to validate the result. The other heuristic methods have test cases only to run the experiment.
%
%\begin{figure}[ht!]
%\centering
%\begin{tikzpicture}
%	\begin{axis}
%		\addplot+[smooth] coordinates
%		{ (1, 114182) 
%		(2, 142542 ) 
%		(3, 218382) 
%		(4, 172671) 
%		(5,  18946) 
%		(6,  77096) 
%		(7, 141706) 
%		(8, 59282) 
%		(9, 165544) 
%		(10, 1094350) 
%		(11, 157322) 
%		(12, 17389) 
%		(13, 133522) 
%		(14, 426552 ) 
%		(15, 121555) 
%		(16, 359984) 
%		(17,  92510) 
%		(18,  174338) 
%		(19, 482968) 
%		(20, 113902) 
%		(21, 191888) 
%		(22, 149623) 
%		(23, 1174258) 
%		(24, 39179) 
%		(25, 70712) 
%		(26, 161571) 
%		(27, 56562) 
%		(28, 2760) 
%		(29, 114192 ) 
%		(30, 41626) 
%		(31, 52594) 
%		(32,  397885) 
%		(33,  451550) 
%		(34, 1073370) 
%		(35, 209575) 
%		(36, 182143) 
%		(37, 41566) 
%		(38, 109685) 
%		(39, 34183) 
%		(40, 419170) 
%		(41, 59050) 
%		(42, 115695) 
%		(43, 23795 ) 
%		(44, 532410) 
%		(45, 82125) 
%		(46,  758) 
%		(47,  314069) 
%		(48, 3541) 
%		(49, 222488) 
%		(50, 70647) 
%		};
%	\end{axis}
%\end{tikzpicture}
%\caption{Residue gotten from $KK$ for 50 instances}
%\end{figure}
%
%\begin{figure}[ht!]
%\centering
%\begin{tikzpicture}
%	\begin{axis}
%		\addplot+[smooth] coordinates
%		{ (1, 81) 
%		(2, 33 ) 
%		(3, 112) 
%		(4, 40) 
%		(5,  46) 
%		(6,  41) 
%		(7, 24) 
%		(8, 7) 
%		(9, 165544) 
%		(10, 1094350) 
%		(11, 157322) 
%		(12, 17389) 
%		(13, 133522) 
%		(14, 426552 ) 
%		(15, 121555) 
%		(16, 359984) 
%		(17,  92510) 
%		(18,  174338) 
%		(19, 482968) 
%		(20, 113902) 
%		(21, 191888) 
%		(22, 149623) 
%		(23, 1174258) 
%		(24, 39179) 
%		(25, 70712) 
%		(26, 161571) 
%		(27, 56562) 
%		(28, 2760) 
%		(29, 114192 ) 
%		(30, 41626) 
%		(31, 52594) 
%		(32,  397885) 
%		(33,  451550) 
%		(34, 1073370) 
%		(35, 209575) 
%		(36, 182143) 
%		(37, 41566) 
%		(38, 109685) 
%		(39, 34183) 
%		(40, 419170) 
%		(41, 59050) 
%		(42, 115695) 
%		(43, 23795 ) 
%		(44, 532410) 
%		(45, 82125) 
%		(46,  758) 
%		(47,  314069) 
%		(48, 3541) 
%		(49, 222488) 
%		(50, 70647) 
%		};
%	\end{axis}
%\end{tikzpicture}
%\caption{Time in milliseconds for $KK$ to work on 50 instances}
%\end{figure}
%

\subsection{Results}

In Table \ref{kkRandomAndHill} we can appreciate that KK algorithm runs faster than any other proposed heuristic, which makes sense considering the simple operations that are done compared to Random Repeated and Hill Climbing which have to compute the residue of each sum at each step.\\
The best residue that $K$ was able to produce is 758 for file 46, for which Random Repeated and Hill Climbing produced terrible residues. The execution time for Random Repeated and Hill Climbing did not change that much.\\
\\
Since Random Repeated does not consider the current state of the solution, we are practically just iterating and hope to be lucky enough to get a combination that would generate a better residue. \\
Hill Climbing also is based on random swapping but it will moving to a better neighbor and iterating for another one might get us stuck into a local optima that is far from the one that we would expect.
\begin{longtable}{|c|cc|cc|cc|}
\caption{Comparing results with KK,  Random Repeated and Hill Climbing\label{kkRandomAndHill}}\\
\hline
 	 & \multicolumn{2}{c}{ KK} 	 & \multicolumn{2}{c}{ Random Repeated} 	 & \multicolumn{2}{c|}{ Hill Climbing} \\
\hline
File	 & Residue 	 & 	 Milliseconds	 & Residue 	 & 	 Milliseconds	 & Residue 	 & 	 Milliseconds\\
\hline
1	 &114182	 & 	92	 &124122290	 & 	513	 &323327088	 & 	645\\
2	 &142542	 & 	51	 &687210366	 & 	397	 &106455160	 & 	276\\
3	 &218382	 & 	52	 &36606166	 & 	269	 &288046066	 & 	256\\
4	 &172671	 & 	18	 &673420835	 & 	268	 &50523635	 & 	245\\
5	 &18946	 & 	43	 &118084310	 & 	260	 &75153670	 & 	265\\
6	 &77096	 & 	40	 &36200566	 & 	298	 &472988564	 & 	251\\
7	 &141706	 & 	17	 &290976336	 & 	279	 &68465818	 & 	255\\
8	 &59282	 & 	14	 &313058340	 & 	407	 &168428118	 & 	229\\
9	 &165544	 & 	60	 &377751230	 & 	341	 &9664978	 & 	253\\
10	 &1094350	 & 	8	 &82735178	 & 	349	 &520089022	 & 	275\\
11	 &157322	 & 	6	 &137124322	 & 	362	 &523831054	 & 	243\\
12	 &17389	 & 	24	 &246837135	 & 	338	 &57352739	 & 	258\\
13	 &133522	 & 	6	 &519902896	 & 	416	 &305093332	 & 	228\\
14	 &426552	 & 	20	 &168683632	 & 	312	 &325835110	 & 	262\\
15	 &121555	 & 	10	 &79513781	 & 	241	 &25971579	 & 	247\\
16	 &359984	 & 	19	 &456108672	 & 	423	 &72090110	 & 	251\\
17	 &92510	 & 	4	 &526761610	 & 	266	 &331717516	 & 	255\\
18	 &174338	 & 	15	 &216579114	 & 	255	 &142002284	 & 	232\\
19	 &482968	 & 	5	 &604771234	 & 	287	 &196379512	 & 	240\\
20	 &113902	 & 	15	 &546133570	 & 	254	 &182198260	 & 	255\\
21	 &191888	 & 	6	 &322730786	 & 	268	 &74419550	 & 	255\\
22	 &149623	 & 	5	 &532788945	 & 	268	 &70517311	 & 	240\\
23	 &1174258	 & 	15	 &446543408	 & 	277	 &359606686	 & 	324\\
24	 &39179	 & 	4	 &75216663	 & 	271	 &632681317	 & 	319\\
25	 &70712	 & 	14	 &176375630	 & 	276	 &93852932	 & 	259\\
26	 &161571	 & 	4	 &41853471	 & 	259	 &1903253797	 & 	248\\
27	 &56562	 & 	16	 &353456552	 & 	263	 &740153880	 & 	299\\
28	 &2760	 & 	4	 &170741138	 & 	284	 &539330572	 & 	280\\
29	 &114192	 & 	4	 &164227170	 & 	267	 &84164316	 & 	281\\
30	 &41626	 & 	38	 &11218250	 & 	264	 &198030510	 & 	271\\
31	 &52594	 & 	3	 &481985390	 & 	249	 &187076024	 & 	238\\
32	 &397885	 & 	9	 &709146039	 & 	286	 &125592209	 & 	254\\
33	 &451550	 & 	4	 &154726820	 & 	266	 &907916102	 & 	262\\
34	 &1073370	 & 	28	 &321383432	 & 	243	 &459510096	 & 	260\\
35	 &209575	 & 	4	 &358859047	 & 	270	 &9427843	 & 	237\\
36	 &182143	 & 	18	 &1336597985	 & 	331	 &312750889	 & 	249\\
37	 &41566	 & 	3	 &263114974	 & 	258	 &727491300	 & 	240\\
38	 &109685	 & 	4	 &63529643	 & 	262	 &38336147	 & 	294\\
39	 &34183	 & 	19	 &169109701	 & 	267	 &542152423	 & 	264\\
40	 &419170	 & 	4	 &382161400	 & 	260	 &625390808	 & 	243\\
41	 &59050	 & 	4	 &142152116	 & 	271	 &821492128	 & 	247\\
42	 &115695	 & 	11	 &535822457	 & 	250	 &111365141	 & 	254\\
43	 &23795	 & 	4	 &208174055	 & 	280	 &784144699	 & 	263\\
44	 &532410	 & 	3	 &296235488	 & 	256	 &556561466	 & 	245\\
45	 &82125	 & 	47	 &87212219	 & 	273	 &224917259	 & 	247\\
46	 &758	 & 	3	 &93110072	 & 	257	 &273460596	 & 	241\\
47	 &314069	 & 	11	 &80326695	 & 	273	 &334603447	 & 	252\\
48	 &3541	 & 	65	 &457741769	 & 	271	 &621652513	 & 	253\\
49	 &222488	 & 	3	 &94748350	 & 	263	 &176140458	 & 	254\\
50	 &70647	 & 	2	 &289828027	 & 	266	 &224036297	 & 	240\\
\hline
\end{longtable}

In Table \ref{kkSimulatedAndMixedRandom}, we can appreciate that also Simulated Annealing did not perform very well in terms of residue. Moving to worse neighbors really impact in the result generating extremely bad residues. \\
Also the running time for Simulated Annealing increases due to the operations that have to be computed to decide if we should move to a bad neighbor or not.\\
The residues really improve when we mixed heuristics. As can be appreciated in the Mixed Random Repeated (MRR) column, the residues are very small compared to KK. The best result was for file 18, with a residue of 4, and for the file 46, the residue was 156 which beats KK in its base case, and every other one.\\
\\
The only inconvenience with MRR is its running time. The running time of MRR is almost as twice of Simulated Annealing and almost as 4 times what Repeated Random and Hill Climbing take. This makes sense considering that Repeated Random executes in $O(|A|)$, since the generation of random indices and swapping takes constant time, the only thing that takes time for Repeated Random is calculating the sum over $A$.\\
For the mixed scenario, we do not sum over $A$, but we apply the algorithm over an array with size in $O(|A|)$, since after prepartitioning we might remove some elements and the array might be smaller.\\
\\
The cost of MRR resides in KK, so it runs in $O(|A| \log |A|)$, which we considered very good according to the residues that it gets. 

\begin{longtable}{|c|cc|cc|cc|}
\caption{Comparing results with KK,  Simulated Annealing and Mixed Random Repeated\label{kkSimulatedAndMixedRandom}}\\
\hline
 	 & \multicolumn{2}{c}{ KK} 	 & \multicolumn{2}{c}{ Simulated Annealing} 	 & \multicolumn{2}{c|}{ Mixed Random Repeated} \\
\hline
File	 & Residue 	 & 	 Milliseconds	 & Residue 	 & 	 Milliseconds	 & Residue 	 & 	 Milliseconds\\
\hline
1	 &114182	 & 	92	 &5628844163290	 & 	886	 &262	 & 	1191\\
2	 &142542	 & 	51	 &7217281758456	 & 	620	 &372	 & 	885\\
3	 &218382	 & 	52	 &2878871546202	 & 	573	 &102	 & 	881\\
4	 &172671	 & 	18	 &666258108611	 & 	610	 &307	 & 	869\\
5	 &18946	 & 	43	 &7632874636326	 & 	590	 &408	 & 	957\\
6	 &77096	 & 	40	 &934620286296	 & 	603	 &410	 & 	1148\\
7	 &141706	 & 	17	 &7162116942232	 & 	582	 &474	 & 	868\\
8	 &59282	 & 	14	 &597396995426	 & 	656	 &38	 & 	835\\
9	 &165544	 & 	60	 &11892733048032	 & 	581	 &6	 & 	852\\
10	 &1094350	 & 	8	 &2301236163138	 & 	579	 &56	 & 	875\\
11	 &157322	 & 	6	 &568843491078	 & 	587	 &24	 & 	842\\
12	 &17389	 & 	24	 &7565861670717	 & 	570	 &53	 & 	879\\
13	 &133522	 & 	6	 &4683919243710	 & 	579	 &82	 & 	856\\
14	 &426552	 & 	20	 &3424946559084	 & 	613	 &478	 & 	849\\
15	 &121555	 & 	10	 &583856455495	 & 	639	 &31	 & 	853\\
16	 &359984	 & 	19	 &9885670576046	 & 	556	 &474	 & 	909\\
17	 &92510	 & 	4	 &3498918646720	 & 	607	 &6	 & 	838\\
18	 &174338	 & 	15	 &4529287452558	 & 	589	 &4	 & 	880\\
19	 &482968	 & 	5	 &237340093792	 & 	560	 &226	 & 	866\\
20	 &113902	 & 	15	 &6422793728994	 & 	597	 &144	 & 	1064\\
21	 &191888	 & 	6	 &8016240362984	 & 	676	 &142	 & 	861\\
22	 &149623	 & 	5	 &499952415977	 & 	564	 &25	 & 	862\\
23	 &1174258	 & 	15	 &4072419307264	 & 	592	 &240	 & 	857\\
24	 &39179	 & 	4	 &2867083895877	 & 	576	 &155	 & 	873\\
25	 &70712	 & 	14	 &130915334860	 & 	591	 &172	 & 	861\\
26	 &161571	 & 	4	 &8096375661177	 & 	586	 &21	 & 	852\\
27	 &56562	 & 	16	 &182986537564	 & 	580	 &390	 & 	867\\
28	 &2760	 & 	4	 &362811874734	 & 	577	 &230	 & 	1005\\
29	 &114192	 & 	4	 &2435593865940	 & 	569	 &642	 & 	918\\
30	 &41626	 & 	38	 &7206058598972	 & 	602	 &138	 & 	822\\
31	 &52594	 & 	3	 &2675618126524	 & 	579	 &200	 & 	926\\
32	 &397885	 & 	9	 &1704894217047	 & 	556	 &131	 & 	1136\\
33	 &451550	 & 	4	 &3683242392588	 & 	579	 &110	 & 	948\\
34	 &1073370	 & 	28	 &4386435495758	 & 	579	 &28	 & 	884\\
35	 &209575	 & 	4	 &4349655004671	 & 	570	 &105	 & 	839\\
36	 &182143	 & 	18	 &13260978395405	 & 	607	 &13	 & 	881\\
37	 &41566	 & 	3	 &9691243560948	 & 	592	 &50	 & 	843\\
38	 &109685	 & 	4	 &5041972691309	 & 	610	 &61	 & 	866\\
39	 &34183	 & 	19	 &891678421687	 & 	764	 &179	 & 	834\\
40	 &419170	 & 	4	 &2776743313208	 & 	629	 &38	 & 	887\\
41	 &59050	 & 	4	 &92442205666	 & 	637	 &360	 & 	830\\
42	 &115695	 & 	11	 &112816335279	 & 	567	 &253	 & 	867\\
43	 &23795	 & 	4	 &3134687481497	 & 	590	 &163	 & 	824\\
44	 &532410	 & 	3	 &1062467313616	 & 	582	 &52	 & 	955\\
45	 &82125	 & 	47	 &1403159039103	 & 	598	 &105	 & 	891\\
46	 &758	 & 	3	 &561503245280	 & 	555	 &156	 & 	876\\
47	 &314069	 & 	11	 &8590215375767	 & 	562	 &195	 & 	872\\
48	 &3541	 & 	65	 &6012728597293	 & 	584	 &73	 & 	836\\
49	 &222488	 & 	3	 &5091469271684	 & 	586	 &56	 & 	877\\
50	 &70647	 & 	2	 &5774852634595	 & 	566	 &191	 & 	856\\
\hline
\end{longtable}

In Table \ref{kkMixedHillMixedSimulated} and based on the results from Table \ref{kkSimulatedAndMixedRandom}, we can appreciate that the mixed heuristics produced betters residues. \\
Among the 3 mixed heuristics, Mixed Repeated Random produced the best results. Mixed Hill Climbing produced better results than simple KK and the simple heuristics but the execution time is close to Mixed Repeated Random but the results are not that good. \\
Still the best result for Mixed Hill Climbing was for file 17, which beat Mixed Random Repeated and Mixed Simulated Annealing. Still Mixed Random Repeated produced better results than Hill Climbing but with the trade-off of running slower.\\
\\
The best result for Mixed Simulated Annealing was file 10, which also beats Mixed Hill Climbing and Mixed Random Repeated. Mixed Hill Climbing and Mixed Simulated Annealing generate kind of similar results but over different files. The only inconvenience with Mixed Simulated Annealing is the running time which is the heuristic that performed poorly, given that it took around 3 times what Mixed Random Repeated took. This might be able to be improved if we simplify the function that determines the probability to move to a bad neighbor, as well as the function T. 

\begin{longtable}{|c|cc|cc|cc|}
\caption{Comparing results with KK,  Mixed Hill Climbing and Mixed Simulated Annealing\label{kkMixedHillMixedSimulated}}\\
\hline
 	 & \multicolumn{2}{c}{ KK} 	 & \multicolumn{2}{c}{ Mixed Hill Climbing} 	 & \multicolumn{2}{c|}{ Mixed Simulated Annealing} \\
\hline
File	 & Residue 	 & 	 Milliseconds	 & Residue 	 & 	 Milliseconds	 & Residue 	 & 	 Milliseconds\\
\hline
1	 &114182	 & 	92	 &600	 & 	887	 &614	 & 	3118\\
2	 &142542	 & 	51	 &1202	 & 	718	 &94	 & 	2935\\
3	 &218382	 & 	52	 &1148	 & 	651	 &636	 & 	2499\\
4	 &172671	 & 	18	 &553	 & 	703	 &261	 & 	2696\\
5	 &18946	 & 	43	 &54	 & 	712	 &204	 & 	2660\\
6	 &77096	 & 	40	 &994	 & 	743	 &186	 & 	2776\\
7	 &141706	 & 	17	 &3034	 & 	719	 &1166	 & 	2816\\
8	 &59282	 & 	14	 &638	 & 	754	 &18	 & 	2757\\
9	 &165544	 & 	60	 &1088	 & 	784	 &282	 & 	2816\\
10	 &1094350	 & 	8	 &566	 & 	678	 &4	 & 	2861\\
11	 &157322	 & 	6	 &1394	 & 	683	 &262	 & 	2804\\
12	 &17389	 & 	24	 &9	 & 	804	 &113	 & 	2723\\
13	 &133522	 & 	6	 &468	 & 	716	 &234	 & 	3171\\
14	 &426552	 & 	20	 &1468	 & 	769	 &8	 & 	3050\\
15	 &121555	 & 	10	 &561	 & 	631	 &33	 & 	2790\\
16	 &359984	 & 	19	 &1170	 & 	793	 &402	 & 	2660\\
17	 &92510	 & 	4	 &4	 & 	678	 &722	 & 	2572\\
18	 &174338	 & 	15	 &328	 & 	988	 &106	 & 	2672\\
19	 &482968	 & 	5	 &244	 & 	847	 &400	 & 	2832\\
20	 &113902	 & 	15	 &66	 & 	987	 &890	 & 	2583\\
21	 &191888	 & 	6	 &76	 & 	696	 &50	 & 	2716\\
22	 &149623	 & 	5	 &4693	 & 	666	 &409	 & 	2707\\
23	 &1174258	 & 	15	 &42	 & 	698	 &22	 & 	2858\\
24	 &39179	 & 	4	 &1189	 & 	675	 &169	 & 	2835\\
25	 &70712	 & 	14	 &476	 & 	683	 &88	 & 	2799\\
26	 &161571	 & 	4	 &245	 & 	721	 &189	 & 	2649\\
27	 &56562	 & 	16	 &466	 & 	704	 &276	 & 	2807\\
28	 &2760	 & 	4	 &18	 & 	697	 &504	 & 	2862\\
29	 &114192	 & 	4	 &20	 & 	774	 &232	 & 	2735\\
30	 &41626	 & 	38	 &2548	 & 	768	 &22	 & 	2743\\
31	 &52594	 & 	3	 &1322	 & 	625	 &214	 & 	2657\\
32	 &397885	 & 	9	 &113	 & 	672	 &131	 & 	2714\\
33	 &451550	 & 	4	 &2858	 & 	672	 &170	 & 	2871\\
34	 &1073370	 & 	28	 &54	 & 	801	 &102	 & 	3247\\
35	 &209575	 & 	4	 &327	 & 	737	 &11	 & 	2883\\
36	 &182143	 & 	18	 &1835	 & 	732	 &119	 & 	2838\\
37	 &41566	 & 	3	 &398	 & 	694	 &18	 & 	2700\\
38	 &109685	 & 	4	 &153	 & 	767	 &333	 & 	3078\\
39	 &34183	 & 	19	 &1259	 & 	765	 &27	 & 	2801\\
40	 &419170	 & 	4	 &58	 & 	749	 &36	 & 	3130\\
41	 &59050	 & 	4	 &1358	 & 	739	 &166	 & 	2651\\
42	 &115695	 & 	11	 &869	 & 	751	 &303	 & 	2814\\
43	 &23795	 & 	4	 &963	 & 	737	 &167	 & 	2816\\
44	 &532410	 & 	3	 &2258	 & 	728	 &774	 & 	2623\\
45	 &82125	 & 	47	 &103	 & 	683	 &261	 & 	2519\\
46	 &758	 & 	3	 &228	 & 	624	 &58	 & 	2853\\
47	 &314069	 & 	11	 &449	 & 	864	 &447	 & 	2800\\
48	 &3541	 & 	65	 &1591	 & 	729	 &285	 & 	2757\\
49	 &222488	 & 	3	 &3194	 & 	726	 &998	 & 	2914\\
50	 &70647	 & 	2	 &183	 & 	630	 &17	 & 	3157\\
\hline
\end{longtable}


\subsection{Mixing Karmarkar-Karp with heuristics}

Discuss briefly how you could use the solution from the Karmarkar-Karp algorithm as a starting point for the randomized algorithms, and suggest what effect that might have. (No experiments are necessary, but feel free to try it.)\\
\\
After we run KK, we already have an upper bound of the residue, that we call $r$. So we take our decisions based on this value.\\
For Random Repeated, in each iteration the elements in $A_1$ and $A_2$ must be sorted. We randomly take one element in $A_1$ and another in $A_2$, if the difference is less than $r$, then stay, otherwise we randomly decide if we move the element from $A_1$ to $A_2$ or the element from $A_2$ to $A_1$. \\
\\
For Hill Climbing, we generate a random solution $A_1$ and $A_2$. If the residue is more than $r$, then we randomly pick an element in $A_1$ and we look for another element in $A_1$ whose difference is less than $r$. When we find it, either element is moved to $A_2$. We do the same with $A_2$. This way we are moving the elements between the sets in a way that we look for improving the current residue $r$ found. \\
\\
For Simulated Annealing, we generate the random solution as in the above version of Hill Climbing and we consider to move to a bad solution depending of how far is the proposed residue from $r$ instead of the current residue.
\subsection{Running the Karkarmar-Karp program}
The program is already compiled and package in the root directory so it can run as follows:

\begin{verbatim}
 java -jar NumberPartition.jar input.txt 
\end{verbatim}

If the user would like to try it in a different environment or recompile it again from scratch, you would only need to run the following two commands:

\begin{verbatim}
mvn clean compile assembly:single
mvn assembly:assembly
\end{verbatim}

After the command second command is executed, it can be killed when running the test cases.\\
\\
The program was tested in the NICE system by connecting through the command:
\begin{verbatim}
ssh cgarciavazquez@nice.fas.harvard.edu
\end{verbatim}

The Java project was developed using Eclipse and Maven. In NumberPartition/src/main/java/edu/hardvard/cscie124/pa3 we are going to find the main classes to run the applications. NumberPartition/src/main/java/edu/hardvard/cscie124/pa3/structure we can find the MapxHeap class which extends from AbstractHeap which implements the Heap interface. AbstractHeap uses internally the PriorityQueue class from Java. The class Solution is used for the Heuristic methods to represent a proposed solution and swap elements between sets $A_1$ and $A_2$, as well as cloning capability and a method to get the sum over a list and a method to get the residue.\\
\\
In NumberPartition/src/main/java/edu/hardvard/cscie124/pa3/heuristic we can find the 7 methods implemented starting with HillClimbing, KarmarkarKarp, MixedHillClimbing, MixedRandomRepeated, MixedSimulatedAnnealing, Random Repeated and SimulatedAnnealing. The class NumberPartitionSolution is used to store each result with its corresponding taken time in milliseconds and residue to display it in the logs as well as output in \LaTeX format.\\
\\
In NumberPartition/src/test/java/edu/hardvard/cscie124/pa3/heuristic we can find the test cases that we uses for the 7 heuristics. This test cases take the 50 input files and apply each heuristic for each set. In NumberPartition/src/test/java/edu/hardvard/cscie124/pa3/ we can find class ElementsInFileGenerator that generates the 50 instances of 100 random positive integers.


\end{document}  
